\section{Modello Base}

\subsection{Elementi del Modello}

\subsubsection{Popolazione}
Distribuzione... 

\subsubsection{Rete stradale}

\subsubsection{Rifugi}
luoghi sicuri dove evacuare. possono essere orizzontali o verticali

\subsubsection{Inondazione dello tsunami}
(time-dependent) scenarios, casualties (wave height Hc) 

\subsection{Evacuazione}
Milling Time....

L'evacuazione può avvenire in due modi: a piedi e in auto. Una volta che ogni agente decide
in che modo evacuare non cambierà scelta per tutta la simulazione.

All'inizio della simulazione ogni agente decide come destinazione il rifugio più vicino 
che verrà raggiunto seguendo il percorso più breve (\it{shortest path}).


\subsubsection{Movimento dei Pedoni}
- descrizione stato dell agente (la propria visione dell ambiente)

- velocità costante di [valore]

Each pedestrian agent is assigned a
walking speed which will not vary over the simulation

The effect of tiredness of evacuees and topography of the
environment have been neglected in this simulation

All evacuees begin to evacuate by
foot to the nearest road.

\subsubsection{Movimento delle Auto}
- descrizione stato dell agente (la propria visione dell ambiente)

- descrizione car following model

\subsection{Limitazioni}

tutte le strade a senso unico con un unica corsia
con limite massimo di velocità di 30mph

non vengono gestite interazioni tra macchine e pedoni.

l´accesso ad un link / le intersezioni non sono gestite.

Nel lavoro [] vengono menzionati turisti e residenti
ma i turisti non vengono gestiti veramente è trattati
con una stessa entità senza un comportamento differenziato.