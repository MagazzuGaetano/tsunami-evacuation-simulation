\section{Introduzione}
Gli tsunami sono eventi naturali pericolosi che negli ultimi decenni 
hanno causato la morte di oltre 250.000 persone e si verificano spesso 
dopo un terremoto.

Rispetto agli altri eventi naturali come uragani, eruzioni vulcaniche e inondazioni, 
i tempi di allerta sono molto più brevi \parencite{katada2006integrated}.
%
Infatti possono raggiungere la costa dopo 20-40 minuti dalla prima scossa oppure dopo ore.
Nel primo caso si parla di near-field tsunami, mentre nel secondo di far-field tsunami. 

Considerando le disastrose conseguenze dovute a questi fenomeni è necessaria un'evacuazione efficiente per salvare vite umane. 

Simulare un'evacuazione in caso di tsunami rappresenta uno strumento molto utile per 
prendere delle contromisure e migliorare le modalità di evacuazione.

I modelli agent-based sono ideali per gestire uno scenario complesso come quello di un'evacuazione da tsunami e
per modellare le diverse dinamiche che emergono quando diversi individui interagiscono tra loro durante un'emergenza. 

Molti modelli non considerano o semplificano alcuni fattori importanti nell'evacuazione da tsunami.
Alcuni modelli limitano i pedoni a muoversi solo lungo la strada e non permettono di sfruttare le aree aperte. 
Altri non gestiscono o riducono in parte le interazioni tra pedoni e altri veicoli nel caso di un'evacuazione multimodale.
%
Spesso non vengono considerati i danni causati dal terremoto prima o durante l'evacuazione.

L'obiettivo di questo lavoro è sviluppare un modello di evacuazione multi-agente in caso di tsunami
di auto e pedoni e gestire le interazioni tra gli agenti nelle intersezioni.
%
Il modello è stato sviluppato usando come base di partenza quello di \textcite{wang2016agent}.

Nella sezione \ref{sec:background} vengono presentati gli elementi principali di un modello di evacuzione da tsunami.
Nella sezione \ref{sec:stato-arte} viene riportato lo stato dell'arte.
%
Nelle sezioni \ref{sec:modello} e \ref{sec:estensione} viene descritto il modello base e la sua estensione e 
nelle sezioni \ref{sec:simulazione} e \ref{sec:analisi} viene mostrata l'impostazione della simulazione e 
l'analisi dei risultati.
%
Infine la conclusione nella sezione \ref{sec:conclusione}.
