\section{Estensione del Modello}
\label{sec:estensione}
In questa sezione verrano evidenziate le limitazioni del modello base e poi descritte le modifiche e le aggiunte effettuate.

Una delle principali limitazioni riguarda le interazioni tra i vari tipi di agenti.
Il modello base considera esclusivamente le interazioni auto-auto
tramite il modello General Motors, e non prevede nessuna interazione pedone-pedone o pedone-auto.

Non vengono considerati i danni che potrebbe casuare il terremoto e i rifugi hanno capacità illimitata.
%
Inoltre viene assunto che tutti gli agenti conoscano il percorso più breve per il rifugio più vicino.

Quindi ci si concentrerà sulle interazioni nelle intersezioni introducendo meccanismi di coordinazione tra i vari tipi di agente.
Inoltre la velocità dei pedoni verrà modificata in base alla congestione in modo da poter rappresentare uno scenario più realistico.

\subsection{Rete Stradale}
Tutte le strade della rete sono state considerate come strade locali secondo il \textcite{seaside2010tsp},
ovvero strade a doppio senso e a una corsia con una larghezza variabile da 7.3 m a 9 m e opzionalmente con
un marciapiedie per ogni lato della strada con una larghezza fissa di 1.5 m.
%
È stato assunto che tutte le strade abbiano marciapiedi su entrambi i lati e che la larghezza sia fissata al valore minimo (7.3 m).

\begin{figure}[ht]
    \centering
    \begin{subfigure}{0.475\textwidth}
        \includegraphics[width=\textwidth]{images/intersections}
        \caption{intersezioni classificate per numero di strade: verde a 4 strade (98 nodi) e grigio a 3 strade (206 nodi).}
        \label{fig:intersections}
    \end{subfigure}
    \hfill
    \begin{subfigure}{0.475\textwidth}
        \includegraphics[width=\textwidth]{images/int_type}
        \caption{intersezioni a 4 strade classificate in base al tipo: AWSC in blu (20 nodi) e TWSC in rosso (78 nodi).}
        \label{fig:intersections_types}
    \end{subfigure}
    \caption{Tipi di intersezioni.}
    \label{fig:ints_map}
\end{figure}

Tramite l'utilizzo di OpenStreetMap\footnote{\url{https://www.openstreetmap.org/relation/186505}} e Google Maps\footnote{\url{https://www.google.com/maps/place/Seaside,+Oregon+97138,+Stati+Uniti}} sono state estratte manualmente le posizioni e le direzioni delle strade con stop
delle intersezioni per la città di Seaside, Oregon e classificate in base alla segnaletica stradale:

\begin{itemize}
    \item All-Way Stop Controlled (AWSC): intersezioni con stop in tutte le strade.
    \item Two-Way Stop Controlled (TWSC): intersezioni a 4 strade con due stop nelle strade secondarie o intersezioni a T (3 strade) con uno stop nella strada secondaria.
\end{itemize}
Tutte le intersezioni presenti nella rete hanno massimo 4 strade e non hanno semafori.

In questo lavoro, per semplicità, sono state gestite solo intersezioni a 4 strade.
Come mostrato nella figura \ref{fig:ints_map}, la rete è composta da 304 intersezioni
di cui 206 a 3 strade e 98 a 4 strade. Tra le intersezioni a 4 strade, 20 sono di tipo AWSC e 78 di tipo TWSC.

\subsection{Gestione delle Intersezioni}
% Per la gestione delle intersezioni sono stati considerati esclusivamente le intersezioni a 4 strade e trattati come intersezioni di tipo AWSC (All Way Stop Controlled) o TWSC (Two Way Stop controlled).
La gestione delle intersezioni consiste nel definire quali auto possano passare rispettando le precedenze di auto e pedoni.
Si assume che i pedoni hanno sempre la precedenza e non hanno alcuna attesa per l'attraversamento.

Per gestire queste interazioni tra pedoni e auto è stata introdotta una zona di attraversamento (Fig. \ref{fig:crossing-area}).

% TODO: anticipare in breve cosa succede quando le auto entrano nella zona?

È stato assunto inoltre che le intersezioni abbiano lunghezza e larghezza pari alla larghezza della strada, ovvero 10.3 m.
Dal momento che la rappresentazione della strada è \textit{network-based} è stato deciso che
la zona di attraversamento inizi prima dell'intersezione e termini dopo dell'intersezione a una distanza pari alla metà della larghezza dal centro (5.15 m).

\begin{figure}[ht]
    \centering
    \includegraphics[width=0.5\textwidth]{images/crossing_area}
    \caption{Zona di attraversamento di un'intersezione.}
    \label{fig:crossing-area}
\end{figure}

\pagebreak

Per rappresentare l'attraversamento dei pedoni e delle auto a ogni intersezione $i$ sono state aggiunte le seguenti informazioni:
\begin{itemize}
    \item $C(i, j)$ = numero di pedoni che stanno usando l'attraversamento pedonale che si trova sulla strada che va dall'intersezione $i$ all'intersezione $j$.
    \item $\textit{Arrival}(i)$ = insieme delle auto entrate nella zona di attraversamento.
    \item $\textit{Crossing}(i)$ = insieme delle auto che possono passare contemporaneamente.
    \item $\textit{Stops}(i)$ = insieme delle intersezioni $j$ collegate all'intersezione $i$ in cui è presente uno stop nella strada $(i, j)$.
\end{itemize}

% Inoltre per i TWSC nelle intersezioni la presenza degli stop viene segnata, mediante una lista con il numero dell'intersezioni interessate.

Per stabilire la posizione all'interno della rete stradale, le precedenze e la direzione della prossima intersezione per ogni agente $x$
vengono definite:
\begin{itemize}
    \item $x_{prev}$ l'intersezione precendente
    \item $x_{curr}$ l'intersezione corrente
    \item $x_{next}$ la prossima intersezione
    \item $x_{side} \in \{ \textit{left}, \textit{right} \}$  il lato di marciapiede (solo pedoni)
    \item $x_{dir} \in \{\textit{left}, \textit{straight}, \textit{right}\}$ la direzione verso $x_{next}$
\end{itemize}

\subsubsection{Pedoni}
% Dato un pedone $x$ si definiscono $x_{prev}$ l'intersezione precendente, $x_{curr}$ l'intersezione corrente e $x_{next}$ la prossima intersezione.

Data la tripla ($x_{prev}$, $x_{curr}$, $x_{next}$) viene associata ad ogni intersezione collegata a $x_{curr}$ la direzione per raggiungerla seguendo il senso orario a partire da $x_{prev}$,
per cui $I_d$ indica l'intersezione nella direzione $d \in \{\textit{origin}, \textit{left}, \textit{straight}, \textit{right}\}$.
%
La direzione associata all'intersezione $x_{next}$ è quella dove è diretto il pedone e viene identificata da $x_{dir}$.

\pagebreak

Quando un pedone $x$ entra nella zona di attraversamento di $x_{curr}$ dall'intersezione $x_{prev}$ e si trova sul marciapiede sul lato $x_{side}$
ha tre direzioni in cui poter andare: \textit{left}, \textit{straight}, \textit{right} (Fig. \ref{fig:pedestria-crossing}):
\begin{enumerate}[label=\alph*)]
    \item Se $x_{dir} = \textit{straight}$ e $\textit{side} = x_{side}$ viene incrementato $C(x_{curr}, I_{\textit{side}})$ di 1.
    \item Se $x_{dir} \neq x_{side} \land x_{dir} \neq \textit{straight}$ viene incrementato $C(x_{curr}, I_{\textit{origin}})$ di 1.
    \item Se  $x_{dir} = x_{side}$ non viene alterato nessun contatore.
\end{enumerate}

Quando il pedone esce dalla zona di attraversamento il contatore corrispondente viene decrementato di 1, e nel caso b
il lato del marciapiede $x_{side}$ viene impostato a quello opposto.

\begin{figure}[ht]
    \centering
    \includegraphics[width=\textwidth]{images/pedestrian_crossing}
    \caption{
        Esempio dei tre casi di attraversamento dal punto di vista del pedone che si trova sul marciapiede sinistro
        a) il pedone si trova sul lato sinistro e attraversa sul link collegato a origin.
        b) il pedone attraversa sul lato sinistro dell'intersezione, quindi sul link collegato a left.
        c) il pedone segue il marciapiede sulla sinistra senza occupare alcun attraversamento pedonale.
    }
    \label{fig:pedestria-crossing}
\end{figure}


\subsubsection{Auto}
Quando un'auto $x$ raggiunge la zona di attraversamento dell'intersezione $i$ viene aggiunta ad $\textit{Arrival}(i)$ e
in base al tipo di intersezione viene schedulata (Sez. \ref{subsubsec:AWSC}, \ref{subsubsec:TWSC}).

Quando $\textit{Crossing}(i)$ è vuoto, le auto in $\textit{Arrival}(i)$ vengono selezionate e aggiunte a $\textit{Crossing}(i)$. In base al tipo di
intersezione viene dato il via libera solo a quelle che possono passare insieme secondo le regole che verranno descritte successivamente (Sez. \ref{subsubsec:AWSC}, \ref{subsubsec:TWSC}).

%
Una volta che un'auto $x$ ottiene il via libera, viene controllato se ci sono pedoni che stanno attraversando le strade
su cui deve passare per attraversare l'intersezione, ovvero $(x_{curr}, x_{next})$ e $(x_{prev}, x_{curr})$.
Nello specifico se $C(x_{curr},x_{prev}) = 0$ l'auto può iniziare ad attraversare e se $C(x_{curr},x_{next}) = 0$
l'auto può uscire dalla zona di attraversamento (Fig. \ref{fig:auto-ped-crossing}) e viene rimossa da $\textit{Arrival}(i)$ e da $\textit{Crossing}(i)$.
Finché sono presenti pedoni in queste due occasioni l'auto resta in attesa.

\begin{figure}[ht]
    \centering
    \includegraphics[width=0.45\textwidth]{images/crossing_auto_ped_crossing}
    \caption{Esempio degli attraversamenti pedonali (in rosso) che vanno controllati
        per poter passare data la tripla ($x_{prev}$, $x_{curr}$, $x_{next}$) di un'auto $x$.}
    \label{fig:auto-ped-crossing}
\end{figure}

\subsubsection{AWSC}
\label{subsubsec:AWSC}
Nelle intersezioni di tipo AWSC la prima auto che arriva ha la precedenza sulle altre e deve aspettare eventuali pedoni come spiegato in precedenza.

La risoluzione delle precedenze avviene tra le auto che arrivano a un'intersezione allo stesso tempo, 
ovvero quelle che hanno il tempo di arrivo minore presenti in $\textit{Arrival}(i)$.
L'auto che ha la destra libera viene considerata provenire da \textit{origin} e dal suo punto di riferimento
vengono identificate le direzioni di provenienza delle altre auto.

Basandosi sulle direzioni di provenienza delle auto e sulle rispettive destinazioni viene verificato
in senso orario (\textit{origin} $\rightarrow$ \textit{left} $\rightarrow$ \textit{straight})
quali auto possono passare contemporaneamente. Queste auto vengono aggiunte a $\textit{Crossing}(i)$.
%
Nelle tabelle \ref{tab:origin-left}, \ref{tab:origin-straight}, e \ref{tab:origin-left-straight} sono
elencati i casi tra due e tre auto in cui queste possono passare insieme.

\begin{table}[p]
    \centering
    \begin{tabular}{l|l|l|}
        \cline{2-3}
                                                                                   & Car1: Origin & Car2: Left \\ \hline
        \multicolumn{1}{|c|}{\multirow{4}{*}{\rotatebox[origin=c]{90}{Direction}}} & Left         & Right      \\ \cline{2-3}
        \multicolumn{1}{|c|}{}                                                     & Right        & Left       \\ \cline{2-3}
        \multicolumn{1}{|c|}{}                                                     & Right        & Right      \\ \cline{2-3}
        \multicolumn{1}{|c|}{}                                                     & Straight     & Right      \\ \hline
    \end{tabular}
    \caption{Tutte le possibili combinazioni di destinazione delle due auto che permettono
        di passare contemporaneamente nel caso in cui arrivano rispettivamente da \textit{origin} e \textit{left}.}
    \label{tab:origin-left}
\end{table}

\begin{table}[p]
    \centering
    \begin{tabular}{l|l|l|}
        \cline{2-3}
                                                                                   & \multicolumn{1}{c|}{Car1: Origin} & \multicolumn{1}{c|}{Car2: Straight} \\ \hline
        \multicolumn{1}{|c|}{\multirow{5}{*}{\rotatebox[origin=c]{90}{Direction}}} & Left                              & Left                                \\ \cline{2-3}
        \multicolumn{1}{|c|}{}                                                     & Right                             & Right                               \\ \cline{2-3}
        \multicolumn{1}{|c|}{}                                                     & Right                             & Straight                            \\ \cline{2-3}
        \multicolumn{1}{|c|}{}                                                     & Straight                          & Right                               \\ \cline{2-3}
        \multicolumn{1}{|c|}{}                                                     & Straight                          & Straight                            \\ \hline
    \end{tabular}
    \caption{Tutte le possibili combinazioni di destinazione delle due auto che permettono
        di passare contemporaneamente nel caso in cui arrivano rispettivamente da \textit{origin} e \textit{straight}.}
    \label{tab:origin-straight}
\end{table}

\begin{table}[p]
    \centering
    \begin{tabular}{l|l|l|l|}
        \cline{2-4}
                                                                                   & \multicolumn{1}{c|}{Car1: Origin} & \multicolumn{1}{c|}{Car2: Left} & Car3: Straight \\ \hline
        \multicolumn{1}{|c|}{\multirow{4}{*}{\rotatebox[origin=c]{90}{Direction}}} & Right                             & Left                            & Right          \\ \cline{2-4}
        \multicolumn{1}{|c|}{}                                                     & Left                              & Right                           & Left           \\ \cline{2-4}
        \multicolumn{1}{|c|}{}                                                     & Right                             & Right                           & Right          \\ \cline{2-4}
        \multicolumn{1}{|c|}{}                                                     & Straight                          & Right                           & Right          \\ \hline
    \end{tabular}
    \caption{Tutte le possibili combinazioni di destinazione delle tre auto che permettono
        di passare contemporaneamente nel caso in cui arrivano rispettivamente da \textit{origin}, \textit{left} e \textit{straight}.}
    \label{tab:origin-left-straight}
\end{table}

%\pagebreak

\subsubsection{TWSC}
\label{subsubsec:TWSC}
Nel caso in cui l'intersezione sia di tipo TWSC vengono identificate due vie: quella principale che ha la precedenza e quella
secondaria dove sono presenti gli stop.

A differenza delle intersezioni AWSC, l'ordine di precedenza non importa poiché possono passare al più due auto alla volta
che si trovano l'una di fronte all'altra.
Quindi viene scelta casualmente come riferimento una delle due auto che viene considerata provenire da \textit{origin}, mentre
l'altra da \textit{straight} e si verifica se le due auto possono passare insieme controllando il caso origin-straight (Tab. \ref{tab:origin-left-straight}).

Quando la via principale sarà libera, ovvero se tutte le auto in $\textit{Crossing}(i)$ provengono solo dalle intersezioni $\textit{Stops}(i)$, verrà gestita nello stesso modo quella secondaria.

\subsection{Velocità dei Pedoni}
La velocità dei pedoni solitamente viene gestita tramite relazioni macroscopiche tra velocità-densità-flusso
espresse tramite un diagramma fondamentale \parencite{nikolic2016probabilistic}.
Per estrarre queste relazioni sono richiesti dei dati empirici che non sempre disponibili, soprattutto nei casi di emergenza.

Non avendo a disposizione dati empirici è stato deciso di usare l'approccio di \textcite{wang2021novel}
che si basa sullo stesso modello di partenza di questo lavoro.

Ogni pedone ha una velocità massima e aggiorna la sua velocità attuale in base alla densità di fronte
seguendo il diagramma velocita-densità (Fig. \ref{fig:modello-esteso-velocita-pedoni}).

\begin{figure}[ht]
    \centering
    \includegraphics[width=\textwidth]{images/modello_esteso_velocità_pedoni.png}
    \caption{Distribuzione della velocità e relazione velocità-densità di \textcite{wang2021novel} applicata al modello base}
    \label{fig:modello-esteso-velocita-pedoni}
\end{figure}

\pagebreak
