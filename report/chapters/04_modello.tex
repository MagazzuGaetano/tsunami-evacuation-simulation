\section{Descrizione del Modello}
Il modello considerato \cite{mostafizi2019agent} é un sistema multi-agente che prevede l'evacuazione della città di Seaside, Oregon in caso di tsunami di auto e pedoni.
%
% TODO: riscrivere
L'evacuazione non include le conseguenze del terremoto che avviene prima dello tsunami e viene considerata iniziare alla fine di questo.
Il modello è basato sull'uso di dati GIS per la distribuzione della popolazione, la rete stradale ed i rifugi.

%
Per la distribuzione della popolazione é stato considerato uno scenario a mezzogiorno di un fine settimana di estate,
che presenta una maggiore concentrazione di residenti sulla spiaggia e nel centro della città
La popolazione sulla costa e nel centro é distribuita normalmente,
mentre quella nella zona residenziale é distribuita uniformemente (Fig. \ref{fig:population}).

\begin{figure}[ht]
  \includegraphics[width=\textwidth]{images/population}
  \caption{Distribuzione della popolazione nello scenario considerato.
    (a) Mostra le aree in cui è distribuita la popolazione, divise nelle tre macro aree: costa, centro, zona residenziale.
    (b) Immagine satellitare con la distribuzione della popolazione.}
  \label{fig:population}
\end{figure}

\subsection{Ambiente}
L'ambiente é composto dalla rete stradale della città con i relativi rifugi e dello tsunami.

% TODO: riscrivere
La rete stradale é rappresentata da un grafo, i cui nodi corrispondono alle intersezioni e gli archi alle strade.
Tutte le strade sono considerate a senso unico, con una sola corsia e con una velocità limite di 55 km/h.

8 delle intersezioni sono marcate come rifugi con capacità illimitata.

Lo tsunami é rappresentato da una griglia discreta, dove ogni cella contiene i valori temporali di altezza delle onde.
I dati usati in questo progetto sono quelli calcolati dal modello di inondazione ComMIT/MOST \cite{titov1997implementation} per la zona di subduzione della Cascadia.



\subsection{Agenti}
La simulazione può prevedere diversi tipi di agenti: residenti, pedoni e auto.

\subsubsection{Residenti}
All'inizio dell'evacuazione i residenti si trovano all'esterno di edifici e auto [vedi parte popolazione]
ed autonomamente scelgono come evacuare. Le scelte sono evacuare a piedi o in auto, verticalmente o orizzontalmente.
Una volta che ogni agente decide in che modo evacuare non cambierà scelta per tutta la simulazione.

All'inizio della simulazione ogni residente sceglie come destinazione il rifugio più vicino raggiungibile.
FINITO IL MILLING TIME si muove verso l'intersezione più vicina e
inizia a seguire il percorso più breve per il rifugio nella MODALITA' scelta (car/foot i mean).

\vspace*{4mm}
Per calcolare i percorsi più brevi é stato usato l'algoritmo A*.

\vspace*{4mm}
Il tempo impiegato per prepararsi all'evacuazione (milling time) é modellato tramite
la distribuzione di Rayleigh (Eq. \ref{eq:rayleigh}), con un tempo minimo ($\tau$) di 10 minuti
e un parametro di scala ($\sigma$) di 1.65.
Questo tempo comprende anche il raggiungimento del veicolo.
%
\begin{equation}
  f(x; \tau, \sigma) = \frac{(x - \tau)^2}{\sigma^2}e^{-{(x - \tau)^2}/(2\sigma^2)}
  \label{eq:rayleigh}
\end{equation}

Gli agenti durante l'evacuazione possono continuare sulla strada attuale o cambiare strada seguendo il percorso,
morire se l'altezza dell'onda nel punto in cui si trova é superiore o uguale a un parametro $H_c$,

\subsubsection{Pedoni}
La velocità di camminata viene stabilita tramite una distribuzione normale
con media 1.21 m/s e deviazione standard 0.20 m/s.
La velocità di ogni pedone rimane costante durante tutta l'evacuazione.

% Non viene gestita alcuna interazione Pedone-Pedone o Pedone-Auto.

\subsubsection{Auto}
Ogni auto conteniene una sola persona per considerare il caso peggiore.
Le auto possono raggiungere la velocità massima imposta dalla strada, ovvero 55 km/h.

Il comportamento delle auto é modellato tramite il modello car-following, General-Motors.

\section{Estensione del Modello}
Il modello é stato esteso andando a gestire le interazioni tra gli agenti nelle intersezioni.

\subsection{Comportamento dei Pedoni}
E' stata aggiunta una variabilità nella velocità di camminata

Marciapiedi
Striscie pedonali

\subsection{Gestione degli Intersezioni}
