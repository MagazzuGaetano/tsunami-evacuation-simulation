\section{Stato dell'arte}
\label{sec:stato-arte}
I primi modelli di evacuazione da tsunami sono stati basati sui modelli network-based utilizzati per l'evacuazione da altri disastri come
uragani, incendi e inondazioni \parencite{usuzawa1997development, imamura2001development}.

% TODO: riscrivere
Uno dei primi aspetti che è stato preso in considerazione è il comportamento umano,
in particolare come i residenti reagiscono all'arrivo dello tsunami
e quanto tempo ci mettono per iniziare ad evacuare.
%
Queste informazioni sono state raccolte tramite dei questionari rivolti ai residenti
e usate per modellare i tempi di partenza dell'evacuazione \parencite{imamura2001development, saito2004simulation}.

Questi primi modelli network-based hanno usato come regola di \textit{path finding}
proseguire verso il prossimo nodo più alto. Successivamente si è passati a usare il percorso
più breve \parencite{katada2004disaster} e altre strategie di routing basate sull'apprendimento come Nash equilibrium e system optimal \parencite{lammel2009towards}.

\vspace*{4mm}
Con l'aumento della potenza di calcolo è stato possibile passare da i modelli network-based ai modelli grid-based e ibridi.
Inoltre è stato possibile usare una quantità di dati maggiore e sfruttare il calcolo parallelo \parencite{wijerathne2013hpc, makinoshima2018enhancing}.

\vspace*{4mm}
Successivamente i modelli di evacuazione da tsunami sono stati sfruttati per prendere delle contromisure e migliorare le modalità di evacuazione,
analizzando le congestioni del traffico, il posizionamento dei rifugi e lo scambio di informazioni.
%TODO: aggiungere citazioni


\newpage
\noindent
Recentemente sempre più lavori hanno analizzato e approfondito diversi aspetti.

/

\noindent
Qua sotto descriviamo alcuni lavori che hanno approfondito.

\vspace*{4mm}
\textbf{\textcite{lammel2010emergency}} Queue model, NASH

In molti modelli vengono considerati solo pedoni oppure pedoni e auto, ma senza gestire le interazioni tra essi.
\textcite{goto2012tsunami} hanno considerato gli individui raggruppati in famiglie e categorizzato le famiglie in:
pedoni lenti, pedoni normali, motociclisti e occupanti di un'auto.
%Inoltre hanno modellato le interazioni dei diversi agenti all'interno di una corsia della strada.
\textcite{wang2021novel} hanno ripreso il lavoro di \textcite{goto2012tsunami} e ...

Un'altro aspetto importante per l'evacuazione è la conoscenza dell'ambiente da parte degli agenti.
Alcuni lavori distinto gli agenti in base alla loro conoscenza e
studiato gli effetti di diverse proporzioni tra categorie di agenti.
\textcite{nguyen2012simulation} hanno definito \textit{fox agent} un pedone ben informato che segue i segnali
stradali fino a un rifugio e \textit{sheep agent} un pedone che non sa
come comportarsi e quindi segue i \textit{fox agent} o si muove casualmente.
\textcite{takabatake2017simulated} invece hanno distinto gli agenti in residenti e visitatori.
I residenti sono agenti che conoscono il percorso più breve per evacuare, mentre i visitatori
seguono gli altri scegliendo la strada con più individui o si muovono verso una zona più elevata.

\textbf{\textcite{mas2012agent}}

\textcite{wijerathne2013hpc} hanno proposto un modello grid-based che utilizza un sistema di navigazione basato
sulla visione. Gli agenti si muovono verso un luogo sicuro ben visibile scegliendo la strada con una maggiore distanza di visione.
%
Anche in questo lavoro vengono distinti visitatori, che si affidano alla visione, e non-visitatori, che hanno conoscenza di un'area
limita al di fuori della quale vengono considerati visitatori.


Altri lavori hanno analizzato come i tempi di partenza influenzano i tempi di evacuazione e le vittime
\textbf{\textcite{wang2016agent}}.

\textbf{\textcite{wang2021novel}} propongono un modello più realistico che considera i danni del terremoto
prima dello tsunami sulla rete stradale.
