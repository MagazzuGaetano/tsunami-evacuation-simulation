\section{Background}
Un modello di evacuazione in caso di tsunami e' composto da diversi elementi.

\subsection{Rete stradale / Griglia}
L'ambiente può essere rappresentato da modelli grid-based o network-based.

I modelli grid-based rappresentano la città come griglia e permettono di muovere gli agenti in due dimensioni 
e sfruttare spazi ampi senza limitarsi alle strade \parencite{makinoshima2018enhancing}.
Inoltre permettono una rappresentazione più efficace delle dinamiche tra gli agenti, ma il cui costo computazionale 
cresce con il numero di agenti considerati.

I modelli network-based invece rappresentano la rete stradale come un grafo dove i nodi sono le intersezioni e gli archi le strade percorribili.
Gli archi possono avere associate delle informazioni come la lunghezza e la larghezza della strada che permettono di stabilire i tempi di percorrenza e la capacità.
Con questa rappresentazione gli agenti possono muoversi solo lungo gli archi, al contrario di quelli grid-based che hanno maggior libertà.
Questi modelli sono piu limitati nella rappresentazione delle interazioni tra agenti, ma risultano più efficienti computazionalmente.

\subsection{Distribuzione della Popolazione}
La distribuzione spaziale della popolazione al momento dell'evacuazione varia in base a diversi fattori:
l'ora del giorno, la stagione, eventi. Inoltre la popolazione può essere composta da residenti e
turisti che si comporteranno diversamente al pericolo.

\subsection{Rifugi}
I rifugi sono dei luoghi sicuri dove evacuare.
Si distinguono in luoghi al di fuori della zona a rischio (rifugi orizzontali) o
luoghi che si trovano ad una altezza che lo tsunami non raggiunge (rifugi verticali).
%
Solitamente si assume che resistano a terremoti e tsunami e
che abbiano una capacità illimitata.

\subsection{Inondazione da Tsunami}
Le inondazioni da tsunami vengono simulate utilizzando un modello di inondazione
che fornisce una serie temporale di altezza e velocità delle onde in una determinata area.

\subsection{Casualty Model}
Un casualty model defisce quando un individuo viene considerato una vittima.
Spesso si utilizza semplicemente l'altezza dello tsunami.
In alcuni studi vengono considerati altri fattori come la velocità
e la temperatura dello tsunami o anche l'età e il genere dell'individuo \parencite{yeh2010gender}.

