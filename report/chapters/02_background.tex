\section{Background}
% MODELLO GENERICO
% \# TODO: descrivere le componenti di un modello generico di simulazione di evacuazione in caso di tsunami..
Il sistema di evacuazione in caso di tsunami e' composto da diversi elementi.

\subsection{Elementi del Modello}

\subsubsection{Distribuzione della Popolazione}
La distribuzione della popolazione al momento dell'evacuazione varia in base a diversi fattori: 
l'ora del giorno, la stagione, eventi. Inoltre la popolazione può essere composta da residenti e 
turisti che si comporteranno diversamente al pericolo. 

Genere, età, ....

? Questa può avere un grosso impatto sull'evacuazione (se i  turisti sono lontani dagli shelter) ?

\subsubsection{Rete stradale (Solo network based però)}

\subsubsection{Rifugi}
I rifugi sono dei luoghi sicuri dove evacuare.
Si distinguono in luoghi al di fuori della zona a rischio o 
luoghi all'interno ad una certa altezza che lo tsunami non raggiunge.

Solitamente si assume che resistano a terremoti e tsunami e che abbiano una capacità illimitata.

\subsubsection{Inondazione da Tsunami}
Le inondazioni da tsunami vengono simulate utilizzando un modello di inondazione 
che fornisce una serie temporale di altezza e velocità delle onde in una determinata area.

\subsubsection{Casualty Model}
Un casualty model defisce quando un individuo viene considerato una vittima. 
Spesso si utilizza semplicemente l'altezza dello tsunami.
In alcuni studi vengono considerati altri fattori come la velocità 
e la temperatura dello tsunami o anche l'età e il genere dell'individuo \cite{yeh2010gender}.

\subsection{Evacuazione}

% QUESTO SOTTO PERÒ VA NELLA DESCRIZIONE DEL NOSTRO MODELLO BASE

Milling Time....

L'evacuazione può avvenire in due modi: a piedi e in auto. Una volta che ogni agente decide
in che modo evacuare non cambierà scelta per tutta la simulazione.

All'inizio della simulazione ogni agente decide come destinazione il rifugio più vicino 
che verrà raggiunto seguendo il percorso più breve (\it{shortest path}).


\subsubsection{Movimento dei Pedoni}
- descrizione stato dell agente (la propria visione dell ambiente)

- velocità costante di [valore]

Each pedestrian agent is assigned a
walking speed which will not vary over the simulation

The effect of tiredness of evacuees and topography of the
environment have been neglected in this simulation

All evacuees begin to evacuate by
foot to the nearest road.

\subsubsection{Movimento delle Auto}
- descrizione stato dell agente (la propria visione dell ambiente)

- descrizione car following model

\subsection{Limitazioni}

tutte le strade a senso unico con un unica corsia
con limite massimo di velocità di 30mph

non vengono gestite interazioni tra macchine e pedoni.

l'accesso ad un link / le intersezioni non sono gestite.

Nel lavoro [] vengono menzionati turisti e residenti
ma i turisti non vengono gestiti veramente è trattati
con una stessa entità senza un comportamento differenziato.
