\section{Simulazione}
\label{sec:simulazione}
Per la simulazione alcuni parametri sono stati fissati, mentre altri parametri sono estratti randomicamente 
o sono stati cambiati manualmente per i diversi esperimenti.

\subsection{Parametri Fissati}
\label{ssec:parametri-fissi}

Parametri per il modello \textit{General Motors}:
\begin{itemize}
    \item \textbf{max speed}: 55 km/h. Velocità massima.
    \item \textbf{safe distance}: 1.8 m. Distanza minima tra le auto.
    \item \textbf{acceleration}: 1.5 m/s². Accelerazione nel caso in cui non è presente un'auto davanti.
    \item \textbf{l}: 2. Esponente di distanza.
    \item \textbf{m}: 0. Esponente di velocità.
    \item \textbf{alpha}: 0.36 km²/h. Coefficiente di sensitività. 
\end{itemize}

\noindent
Parametri per il \textit{casualty model}:
\begin{itemize}
  \item \textbf{Hc}: 1 m. Profondità critica alla quale l'agente viene considerato una vittima.
  \item \textbf{Tc}: 120 s. Tempo in acqua che l'agente deve trascorrere per venire considerato una vittima. 
\end{itemize}

\noindent
Parametri per il tempo di preparazione:
\begin{itemize}
  \item \textbf{Rtau}: 10 min. Tempo minimo per prepararsi prima di evacuare.
  \item \textbf{Rsig}: 1.65. Fattore di scala per la distribuzione di Rayleigh.
\end{itemize}

\noindent
Parametri per il comportamento dei pedoni:
\begin{itemize}
  \item \textbf{search\_length}: 4 m. Lunghezza della strada davanti al pedone per calcolare la densità.
  \item \textbf{jam\_density}: 6 ped/m². Densità massima.
\end{itemize}

\noindent
Parametri per le dimensioni delle strade:
\begin{itemize}
  \item \textbf{side\_width}: 1.5 m. Larghezza di un marciapiede.
  \item \textbf{lane\_width}: 3.6 m. Larghezza di una corsia.
\end{itemize}

\subsection{Parametri Variabili}

Parametri per la percentuale di auto e pedoni:
\begin{itemize}
  \item \textbf{R1\_Evac\_Foot}: Probabilità di un residente di evacuare a piedi.
  \item \textbf{R2\_Evac\_Car}: Probabilità di un residente di evacuare in auto.
\end{itemize}

\noindent
Sono state provate le seguenti combinazioni di auto e pedoni:
\begin{itemize}
  \item \textbf{R1\_Evac\_Foot}: 0\%, \textbf{R2\_Evac\_Car}: 100\%
  \item \textbf{R1\_Evac\_Foot}: 25\%, \textbf{R2\_Evac\_Car}: 75\%
  \item \textbf{R1\_Evac\_Foot}: 50\%, \textbf{R2\_Evac\_Car}: 50\%
  \item \textbf{R1\_Evac\_Foot}: 75\%, \textbf{R2\_Evac\_Car}: 25\%
  \item \textbf{R1\_Evac\_Foot}: 100\%, \textbf{R2\_Evac\_Car}: 0\%
\end{itemize}

\noindent
Parametri per il comportamento dei pedoni:
\begin{itemize}
  \item \textbf{$\mu_p$}: media della velocità, distribuita uniformemente in [1.4, 2].
  \item \textbf{$\sigma_p$}: deviazione standard della velocità, distribuita uniformemente in [0.1, 0.6].
  \item \textbf{side}: lato del marciapiede all'inizio della simulazione (0 sinistro e 1 destro), assegnato in modo equiprobabile. 
  Durante la simulazione può variare. 
\end{itemize}

\noindent
Per via della randomicità dei parametri i risultati degli esperimenti sono stati mediati su 30 simulazioni.
