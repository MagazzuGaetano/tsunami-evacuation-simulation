\section{Conclusioni}
\label{sec:conclusione}
In questo lavoro è stato implementato un modello agent-based per la simulazione di evacuazione in caso di tsunami.
Il modello di \textcite{wang2016agent} è stato usato come base di partenza ed esteso aggiungendo una velocità
variabile ai pedoni e la gestione delle interazioni tra auto e pedoni nelle intersezioni.

Successivamente sono stati effettuati diversi esperimenti al variare del numero di auto e pedoni.
I risultati di ogni esperimento sono stati mediati tra 30 simulazioni per via della randomicità di alcuni parametri
e sono stati confrontati con il modello di partenza e in seguito con il modello di \textcite{wang2021novel}.

Nonostante l'evacuazione sia immediata non tutti riescono ad evacuare nei casi con traffico misto.

La percentuale di evacuati aumenta al diminuire del numero di auto, quindi il caso migliore risulta essere quello con solo pedoni
con il 99\% di evacuati.

Comparato al modello base il modello esteso ha risultati meno ottimistici.

Per quanto riguarda i pedoni non sembra esserci alcun cambiamento significativo,
infatti si hanno delle percentuali di vittime, evacuati e tempi di evacuazione molto simili con il modello base.

La gestione delle intersezioni rallenta l'evacuazione poichè aggiunge dei tempi di attesa nelle intersezioni,
abbassa il flusso sia in entrata che in uscita e di conseguenza si ha un numero minore di evacuati e
un numero maggiore di vittime rispetto al modello base, soprattutto nei casi con sia auto che pedoni.

I casi con solo auto e solo pedoni sono quelli subiscono meno l'effetto della gestione delle intersezioni.

Infine comparato al modello di \textcite{wang2021novel} il modello base e il modello esteso hanno risulti simili per quanto riguarda i pedoni, nonostante
molti fattori tra i modelli siano diversi come ad esempio i tempi di partenza, la distribuzione di velocità e l'interazione con le auto.

Per le auto invece si ha una differenza significativa dovuta alla numerosità degli agenti, poichè viene considerato che un auto corrisponde a 4 persone, e alla 
velocità massima considerata. 

In particolare il modello esteso risulta più ottimistico per evacuati e vittime rispetto al modello di \textcite{wang2021novel} nonostante la gestione delle intersezioni,
che prevede un rallentamento del traffico e più vittime rispetto al modello base, ma va tenuto conto dell'evacuzione immediata.

\subsection{Sviluppi Futuri}
In questo lavoro ci si è limitati a intersezioni a 4 strade, esclusivamente di tipo AWSC e TWSC, quindi potrebbero essere considerate
ulteriori tipi di intersezioni come ad esempio quelle con 3 strade oppure quelle regolate da semafori.

Schedulare le auto tramite il tempo di arrivo minimo prima del aver effettuato il controllo dei pedoni,
può portare a situazioni in cui delle auto potrebbero passare, ma invece rimangono ferme.
Questo potrebbe essere risolto aggiornando i tempi di arrivo delle auto in attesa e anticipare il controllo dei pedoni prima
selezionare le auto con tempo di arrivo più basso.

Le auto nelle strade secondarie degli incroci di tipo TWSC tendono a prendersi la precendenza alle auto sulla strada principale 
poichè le auto non sono considerate in Arrival finchè non sono sulla linea di stop.

Nelle intersezioni i pedoni hanno sempre la precendenza, in un lavoro futuro si potrebbe pensare di aggiungere delle attese anche per i pedoni
nel caso delle auto stiano già attraversando.

Un'altra possibile aggiunta è quella di introdurre una fase di decelerazione prima di raggiungere la zona di attraversamento e una fase
di accelerazione quando si ottiene il via libera.

Un'altra limitazione è l'utilizzo del percorso più breve, altre strategie di \textit{routing} più realistiche potrebbero essere usate come ad esempio Nash equilibrium.
Per i percorsi dei pedoni inoltre si potrebbe creare un secondo grafo per ottimizzare i percorsi tramite i marciapiedi.

Altri sviluppi futuri suggeriti in altri lavori riguardano considerare scenari a diverse ore del giorno, rendere 
più realistiche le interazioni tra agenti, in casi di congestione abbandonare i veicoli per proseguire a piedi e 
simulare inoltre la propagazione delle comunicazioni durante l'evacuazione e la validazione con dati di precedenti tsunami.
